% Suggested packages/settings (preamble):
% \usepackage{natbib}
% \bibliographystyle{apalike} % or any author-year style you prefer

\section{Introduction}
A large empirical literature documents that stock prices respond to earnings news with delay.
In their seminal event-study analysis, \citet{BallBrown1968} show that accounting earnings contain
information that is impounded into prices around public announcement dates, but that the full
price response is not confined to the announcement window.
Building on this foundation, \citet{BernardThomas1989} formalize and sharpen what became known
as post-earnings-announcement drift (PEAD): after firms report positive (negative) earnings
surprises, cumulative abnormal returns continue to drift upward (downward) over subsequent weeks
and months.\footnote{\citet{BernardThomas1989} emphasize both the magnitude and persistence of the
drift, and frame the core identification question as whether the pattern reflects delayed price response
or compensation for risk; see their discussion of the empirical drift pattern and competing explanations.} %
\citep{BernardThomas1989}. \textit{PEAD is therefore not primarily a question of whether earnings
are informative, but why prices appear to incorporate that information slowly and heterogeneously
across firms and events.}

This paper brings a market microstructure lens to that heterogeneity. Microstructure models treat
order flow as the key state variable through which private information and liquidity demands are
transmitted into prices. In \citet{Kyle1985}, market makers do not observe whether a trade originates
from an informed trader or from noise traders; prices are set as conditional expectations given the
history of aggregate quantities, so price innovations are tied to order-flow innovations \citep{Kyle1985}. %
\citet{GlostenMilgrom1985} formalize the adverse-selection channel in a specialist market:
a bid--ask spread can arise even absent inventory costs because liquidity providers must be compensated
for losses to better-informed traders \citep{GlostenMilgrom1985}. In this environment, transaction prices
form a martingale relative to the specialist's information, so short-horizon return dynamics depend
critically on whether observed price changes reflect information revelation or purely mechanical trading frictions
\citep{GlostenMilgrom1985}. %
Finally, \citet{Hasbrouck1991} provides an econometric implementation that explicitly separates
\emph{transient} price effects (e.g., inventory control and other non-informational frictions) from
\emph{persistent} effects that represent information being permanently impounded into prices:
the information component of a trade is defined as the \emph{ultimate} price impact of the unexpected
component of order flow \citep{Hasbrouck1991}.

These microstructure insights suggest a natural connection to PEAD.
If (part of) earnings-related information diffuses before the announcement---through informed trading,
selective disclosure, or gradual learning---then pre-announcement trading should leave a distinct
microstructure signature: sustained directional order flow whose price impact is largely \emph{permanent}.
Conversely, if pre-announcement trading is dominated by liquidity shocks---indexing flows, hedging,
fund flows, or inventory rebalancing---then order flow may move prices temporarily but should
revert as liquidity conditions normalize, producing \emph{transient} pressure. %
This distinction is closely aligned with Hasbrouck's conceptual separation between persistent
(information) and transient (liquidity/frictions) components of price impact \citep{Hasbrouck1991},
and with the adverse-selection logic of spreads in \citet{GlostenMilgrom1985}.
It also clarifies how one can reconcile strong PEAD in some events with weak or absent PEAD in others:
cross-sectional drift may depend not only on the magnitude of the earnings surprise, but on whether
information about that surprise had already been partially incorporated through informed trading
(or, alternatively, whether trading-induced price pressure in the pre-window was primarily non-informational).

Despite this conceptual fit, the PEAD literature and microstructure literature are often empirically
separated. PEAD studies typically parameterize news with an earnings-based signal such as standardized
unexpected earnings (SUE) and study post-announcement return continuation \citep{BernardThomas1989}.
Microstructure studies, in contrast, focus on how trades move prices (and how liquidity providers respond),
often at intraday frequencies, with less emphasis on the cross-sectional heterogeneity of predictable
post-event returns around known fundamental information events \citep{Kyle1985,GlostenMilgrom1985,Hasbrouck1991}. %
A smaller set of papers examines order imbalance and return predictability, but generally outside the
specific goal of explaining heterogeneity in PEAD conditional on earnings news. For example,
\citet{ChordiaSubrahmanyam2004} show that lagged order imbalances predict short-horizon returns in ways
consistent with autocorrelated trading demands and gradual accommodation by liquidity providers,
with stronger effects in smaller firms \citep{ChordiaSubrahmanyam2004}. This evidence is suggestive:
the same persistence-versus-reversal logic central to inventory and adverse-selection models may matter
for how prices adjust around earnings events, yet existing work does not directly operationalize a
\emph{pre-event} microstructure-based measure that can explain \emph{cross-sectional} PEAD variation
beyond standard earnings-surprise measures.

We address this gap by constructing a continuous, reduced-form \emph{pressure score} from intraday
TAQ data in a fixed pre-announcement window. The score is designed to capture the relative intensity
of information-driven versus liquidity-driven trading pressure prior to earnings announcements. %
Motivated by \citet{Hasbrouck1991}, we build the score from microstructure signatures that proxy for
(1) \emph{persistence} in signed order flow and its price impact (information-like) versus
(2) \emph{burstiness} and liquidity deterioration such as spread widening contemporaneous with flow
(liquidity shock / forced trading). The core output is an event-level scalar that varies continuously
across earnings events.

Empirically, we test whether this pressure score predicts heterogeneity in post-earnings return dynamics
\emph{conditional} on earnings surprise. In other words, we ask whether microstructure-implied
information intensity in the pre-window explains incremental variation in PEAD beyond SUE-based
portfolio sorts. This design naturally admits two complementary interpretations.
First, if the score reflects \emph{pre-leakage} or early information incorporation (in the spirit of
\citet{Kyle1985} and \citet{Hasbrouck1991}), then high information pressure should be associated with
weaker subsequent drift because more information has already been incorporated before the event.
Second, if the score reflects \emph{how} information is incorporated (patient informed trading versus
liquidity-driven dislocations), then information pressure may predict \emph{stronger} drift if
information is only partially incorporated prior to the announcement and continues to diffuse afterward,
consistent with the delayed-response interpretation emphasized in the PEAD debate \citep{BernardThomas1989}.
Either way, the empirical object of interest is the \emph{interaction} between a fundamental news measure
(SUE) and microstructure-based pressure, bringing the two literatures into a single event-study framework.

\paragraph{Scope and positioning.}
We do not attempt a structural identification of information versus liquidity components in the strict
sense of estimating a full trade/quote system and attributing variance shares as in Hasbrouck-style VARs.
Instead, we construct a reduced-form microstructure score designed to capture \emph{relative information
intensity} prior to known information events. This choice is disciplined by microstructure theory
(\citealp{Kyle1985}; \citealp{GlostenMilgrom1985}; \citealp{Hasbrouck1991}) but tailored to the empirical
constraints and objective of explaining cross-sectional PEAD heterogeneity. The goal is a practical
and interpretable bridge: a mapping from high-frequency trading signatures to predictable post-event
return dynamics in the earnings setting.
