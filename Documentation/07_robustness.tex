%========================================================
% Section 7: Robustness
%========================================================

\section{Robustness}

This section evaluates whether the pressure score remains informative under
alternative inference choices and after controlling for earnings-surprise
information. All robustness tests are implemented in
\texttt{Methods\_Sung\_Cho/run\_continuous\_regression.py} and
\texttt{Methods\_Sung\_Cho/run\_continuous\_regression\_with\_sue.py}, with
outputs in
\texttt{data/results/continuous\_regression\_summary.csv} and
\texttt{data/results/continuous\_regression\_with\_sue.csv}.

\subsection{Continuous Regressions with Alternative Clustered Inference}

For each construction method $m\in\{A,B,C,D\}$, we run the pooled event-level
regression
\begin{equation}
R^{(20)}_i = \alpha_m + \beta_m P^{(m)}_i + \varepsilon_i,
\label{eq:robust_baseline_raw}
\end{equation}
where $R^{(20)}_i$ is the 20-trading-day forward cumulative return and
$P^{(m)}_i$ is the method-specific pressure score. The code also estimates the
standardized effect
\begin{equation}
\widetilde P^{(m)}_i = \frac{P^{(m)}_i-\bar P^{(m)}}{s(P^{(m)})},
\qquad
R^{(20)}_i = \alpha_m + \beta^{\text{std}}_m\widetilde P^{(m)}_i + \varepsilon_i,
\label{eq:robust_baseline_std}
\end{equation}
so $\beta^{\text{std}}_m$ is the return change from a one-standard-deviation
increase in pressure.

To ensure inference is not driven by a single dependence structure, we report
three covariance estimators for each $\beta$:
\begin{equation}
\widehat{\mathrm{Var}}_c(\hat\beta)
=(X'X)^{-1}\left(\sum_{g=1}^{G_c} X_g'\hat u_g\hat u_g'X_g\right)(X'X)^{-1},
\end{equation}
with clustering dimension $c\in\{\text{firm},\text{date}\}$, and the
Cameron--Gelbach--Miller two-way combination
\begin{equation}
\widehat{\mathrm{Var}}_{2\mathrm{w}}(\hat\beta)
=\widehat{\mathrm{Var}}_{\text{firm}}(\hat\beta)
+\widehat{\mathrm{Var}}_{\text{date}}(\hat\beta)
-\widehat{\mathrm{Var}}_{\text{firm}\times\text{date}}(\hat\beta),
\label{eq:cgm_vcov}
\end{equation}
where the intersection cluster uses the combined label
$(\text{permno},\text{event\_date})$. The reported $t$-statistic is
$t=\hat\beta/\widehat{\mathrm{se}}(\hat\beta)$.

\begin{table}[!t]
\centering
\caption{Baseline pressure regression robustness (20-day forward return).
$N=4{,}147$ events for each method.}
\label{tab:robust_baseline}
\begin{tabular}{lrrrrrr}
\hline
Method & $\beta^{\text{std}}$ & $t_{\text{firm}}$ & $t_{\text{date}}$ & $t_{2\mathrm{w}}$ & $R^2$ \\
\hline
A & -0.00413 & -2.323 & -2.341 & -2.236 & 0.00155 \\
B & -0.00407 & -2.011 & -2.415 & -2.401 & 0.00151 \\
C & -0.00020 & -0.117 & -0.118 & -0.123 & 0.00000 \\
D & -0.00595 & -3.321 & -3.130 & -3.162 & 0.00322 \\
\hline
\end{tabular}
\end{table}

Table~\ref{tab:robust_baseline} shows that methods A, B, and D produce a
negative pressure-return slope under all three covariance estimators, while
method C is economically and statistically negligible. Method D is strongest
($\beta^{\text{std}}=-0.00595$, $t_{2\mathrm{w}}=-3.162$).

\subsection{SUE-Control Robustness and Interaction Test}

We next test whether the pressure signal survives when controlling for
standardized unexpected earnings (SUE). SUE is built from Compustat quarterly
EPS (\texttt{epsfxq}) using the seasonal-random-walk construction
\citep{BernardThomas1989}:
\begin{align}
\Delta \mathrm{EPS}_{f,q} &\equiv \mathrm{EPS}_{f,q}-\mathrm{EPS}_{f,q-4},\\
\mathrm{SUE}_{f,q} &\equiv
\frac{\Delta \mathrm{EPS}_{f,q}}{\mathrm{sd}\!\left(\Delta \mathrm{EPS}_{f,q-7:q}\right)}.
\end{align}
In implementation, the denominator is an 8-quarter rolling standard deviation
(with minimum 4 observations), and raw SUE is winsorized to $[-10,10]$ before
standardization:
\begin{equation}
\widetilde{\mathrm{SUE}}_i
=\frac{\mathrm{SUE}_i-\overline{\mathrm{SUE}}}{s(\mathrm{SUE})}.
\end{equation}

On the matched sample ($N=4{,}130$), we estimate
\begin{align}
\text{Spec 1: }\;
R^{(20)}_i
&= \alpha + \beta_1\widetilde P_i + \beta_2\widetilde{\mathrm{SUE}}_i + \varepsilon_i,
\label{eq:sue_spec1}\\
\text{Spec 2: }\;
R^{(20)}_i
&= \alpha + \beta_1\widetilde P_i + \beta_2\widetilde{\mathrm{SUE}}_i
+ \beta_3\left(\widetilde P_i\times\widetilde{\mathrm{SUE}}_i\right) + \varepsilon_i,
\label{eq:sue_spec2}
\end{align}
with the same firm, date, and two-way clustered inference as in
Eq.~\eqref{eq:cgm_vcov}.

\begin{table}[!t]
\centering
\caption{SUE-control robustness using two-way clustered (CGM) $t$-statistics.
$N=4{,}130$ events for all rows.}
\label{tab:robust_sue}
\small
\begin{tabular}{llrrrrrr}
\hline
Method & Spec & $\beta_P$ & $t_P$ & $\beta_{\mathrm{SUE}}$ & $t_{\mathrm{SUE}}$ & $\beta_{P\times S}$ & $t_{P\times S}$ \\
\hline
A & 1 & -0.00347 & -1.896 & 0.01055 & 5.919 & -- & -- \\
A & 2 & -0.00348 & -1.898 & 0.01023 & 5.679 & -0.00311 & -2.033 \\
B & 1 & -0.00423 & -2.488 & 0.01077 & 5.669 & -- & -- \\
B & 2 & -0.00405 & -2.364 & 0.01089 & 5.723 & -0.00456 & -3.042 \\
C & 1 & -0.00063 & -0.372 & 0.01074 & 5.614 & -- & -- \\
C & 2 & -0.00060 & -0.353 & 0.01089 & 5.842 & -0.00153 & -1.204 \\
D & 1 & -0.00619 & -3.334 & 0.01083 & 6.133 & -- & -- \\
D & 2 & -0.00615 & -3.307 & 0.01091 & 6.210 & -0.00394 & -2.513 \\
\hline
\end{tabular}
\end{table}

Three conclusions follow from Table~\ref{tab:robust_sue}. First, the SUE term is
strongly positive in every method ($t\approx 5.6$ to $6.6$ across clustering
choices), confirming that earnings-surprise information is priced in the sample.
Second, pressure remains negative in all specifications, and remains clearly
significant for methods B and D after SUE controls under two-way clustering;
method A is borderline and method C remains weak. Third, the interaction term is
negative and significant for methods A, B, and D, implying that the return
sensitivity to pressure becomes more negative when earnings surprise is higher.

\subsection{Overall Robustness Assessment}

Across both robustness exercises, the ranking of pressure constructions is
stable: method D is strongest, methods A and B are informative but weaker, and
method C is not robustly predictive. The key result is that the pressure score
is not an artifact of a specific standard-error estimator and is not fully
subsumed by a standard earnings-surprise control.
