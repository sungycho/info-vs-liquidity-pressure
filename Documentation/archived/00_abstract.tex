\begin{abstract}
This paper studies whether pre-announcement trading pressure helps explain cross-sectional
variation in post-earnings-announcement drift (PEAD). While the PEAD literature
documents that stock returns continue to drift following earnings surprises
\citep{BernardThomas1989}, it typically measures news using accounting-based
signals and abstracts from the microstructure environment in which prices adjust.
In parallel, market microstructure models show that order flow conveys private
information and that trades can have either persistent (informational) or
transient (liquidity-driven) price impacts \citep{Kyle1985,Hasbrouck1991}.
We bridge these literatures by constructing a continuous, event-level
\emph{pressure score} from intraday TAQ data in the pre-announcement window,
designed to proxy for the relative intensity of information-driven versus
liquidity-driven trading.

We test whether this microstructure-based score predicts heterogeneity in
post-earnings returns conditional on standardized unexpected earnings (SUE).
If pre-announcement order flow reflects early information incorporation,
subsequent drift should be attenuated; alternatively, if trading pressure
reflects incomplete information diffusion, drift may be amplified.
Our design does not impose structural identification of information and liquidity
components, but instead implements a reduced-form measure grounded in
microstructure theory and tailored to the event-study setting.
The results speak to how high-frequency trading signatures interact with
fundamental news to shape predictable return dynamics.
\end{abstract}
