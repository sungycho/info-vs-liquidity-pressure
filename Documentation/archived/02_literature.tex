
\section{Literature Review}

\subsection{PEAD and the information content of earnings}
The earnings-announcement literature begins with the recognition that accounting earnings convey
value-relevant information. \citet{BallBrown1968} provide early evidence that stock prices react around
earnings announcement dates, and they develop a prototype event-study methodology that traces abnormal
performance indices around announcements using CRSP returns and announcement dates \citep{BallBrown1968}.
Their findings establish that public earnings releases are meaningful information events and motivate
later work that focuses on the \emph{dynamics} of the price response rather than its existence.

Post-earnings-announcement drift is most prominently documented and analyzed by \citet{BernardThomas1989}.
They show that abnormal returns continue to drift after earnings announcements in a manner monotonic
in earnings surprise deciles, and frame the central interpretive debate as delayed price response versus
risk compensation \citep{BernardThomas1989}. A key methodological theme is that inference about
predictability is sensitive to portfolio formation and risk adjustment. This concern matters for our
setting because a microstructure-based signal could mechanically correlate with liquidity, size, or
trading frictions that confound abnormal-return measurement. Accordingly, our empirical design treats
earnings surprise as the baseline news variable and asks whether microstructure-implied pressure adds
incremental explanatory power for post-event returns beyond standard PEAD controls and specifications.

\subsection{Market microstructure and information in order flow}
Microstructure theory provides a language for distinguishing information revelation from liquidity demand.
In \citet{GlostenMilgrom1985}, a competitive specialist sets bid and ask prices facing heterogeneously
informed traders; the bid--ask spread can arise purely from adverse selection, and transaction prices
form a martingale with respect to the specialist's information \citep{GlostenMilgrom1985}. This framework
motivates using spreads and quote dynamics as proxies for information asymmetry: when the probability of
trading against informed traders rises, liquidity providers protect themselves by widening spreads and
adjusting quotes more aggressively.

\citet{Kyle1985} offers a complementary perspective in a dynamic auction setting with a strategic insider,
noise traders, and competitive market makers. The market maker observes only aggregate order flow, so
price changes are driven by innovations in order flow, and informed traders optimally trade gradually to
hide within noise \citep{Kyle1985}. This gradualism is central for our project: if informed traders split
orders over time, pre-announcement order flow may be persistent and directional without necessarily
creating immediate, easily detectable price jumps. Therefore, microstructure signatures of information
pressure should emphasize persistence and the mapping from signed flow to price changes, rather than
isolated bursts of volume alone.

\citet{Hasbrouck1991} bridges theory and empirics by proposing a formal decomposition of trade impact into
transient and persistent components. Inventory-control and other non-informational effects are inherently
transient, while asymmetric-information effects are permanently impounded in prices; thus, the information
impact of a trade is best measured by the \emph{persistent} price impact of the \emph{unexpected} component
of order flow \citep{Hasbrouck1991}. The emphasis on innovations (unanticipated trades) is directly relevant
to our construction: a predictable component of flow can reflect routine liquidity provision or mechanical
trading patterns, whereas unexpected, persistent directional flow is more plausibly linked to information.

\subsection{Order-flow predictability and informed trading metrics}
A closely related empirical literature studies how order imbalances predict returns and how this relation
depends on liquidity provision. \citet{ChordiaSubrahmanyam2004} analyze daily order imbalances and show that
lagged imbalances predict daily returns, consistent with autocorrelated trading demands and gradual market
maker accommodation; they also emphasize stronger effects in smaller firms where markets absorb imbalances
less expeditiously \citep{ChordiaSubrahmanyam2004}. This evidence is informative for our setting because it
suggests a practical, reduced-form route to distinguish informational from non-informational pressure:
persistent imbalances coupled with stable liquidity conditions can generate continuation, whereas imbalances
that coincide with liquidity deterioration (e.g., spread widening) may be more consistent with inventory or
forced-trading channels that reverse.

An alternative family of approaches seeks to infer informed trading intensity more directly from trade data.
The probability of informed trading (PIN) literature models trading as a game between liquidity providers
and traders, with latent information events affecting the composition of order flow. Building on this line,
\citet{EasleyLopezdePradoOHara2012} propose VPIN, a volume-synchronized measure of order-flow ``toxicity''
intended to track the risk that liquidity providers face from informed trading and to forecast periods of
heightened adverse selection and volatility \citep{EasleyLopezdePradoOHara2012}. While VPIN is not our primary
object, it reinforces a central theme: information risk manifests through the \emph{interaction} of signed
flow with liquidity provision, and market makers respond endogenously when they suspect informed trading.

\paragraph{Positioning of the present paper.}
We do not attempt structural identification of information versus liquidity (as in Hasbrouck-style VAR
decompositions of permanent and transitory components). Instead, we build a reduced-form microstructure
score designed to capture \emph{relative information intensity} prior to known information events. The score
is motivated by (i) adverse-selection and liquidity-provider response (\citealp{GlostenMilgrom1985}),
(ii) gradual informed trading embedded in noise (\citealp{Kyle1985}), and (iii) the idea that information
effects are measured by persistent impacts of unexpected flow (\citealp{Hasbrouck1991}). We then bring this
microstructure measure into the PEAD setting (\citealp{BallBrown1968}; \citealp{BernardThomas1989}) to test
whether pre-event trading pressure helps explain cross-sectional variation in post-earnings return drift
beyond earnings-surprise measures.