%========================================================
% Section 4: Microstructure-Based Pressure Measure
%========================================================

\section{Methodology}

\subsection{Empirical Setting and Event Definition}

The objective of this study is to decompose short-horizon price pressure into
economically distinct components—\emph{information-driven} and
\emph{liquidity-driven}—using market microstructure data. The empirical analysis
is conducted in an event-study framework centered on corporate earnings
announcements, which provide well-defined information events with well-documented
post-announcement return dynamics \citep{BallBrown1968,BernardThomas1989}.

For each earnings announcement, we define a pre-event measurement window of
trading days $[-10,-1]$ relative to the announcement date. All microstructure
features are constructed exclusively within this window. Post-event returns are
used solely for evaluation and play no role in feature construction or pressure
classification.

\subsection{Data Preprocessing}

\subsubsection{Trade--Quote Cleaning and Alignment}

Trades and quotes are cleaned using standard market microstructure filters
\citep{Hasbrouck2007}. Specifically, we remove trades with abnormal condition
codes, exclude quotes with zero or negative spreads, and eliminate locked or
crossed markets. Each trade is aligned to the prevailing NBBO using the most
recent valid quote prior to execution. This alignment is required for accurate
spread measurement and trade classification.

\subsection{Trade Classification and Order Flow Imbalance}

\subsubsection{Lee--Ready Trade Classification}

Trade direction is inferred using the Lee--Ready (1991) algorithm. Let $P_t$
denote the trade price and $M_t = (A_t + B_t)/2$ the contemporaneous midpoint of
the bid $B_t$ and ask $A_t$.

Trades are classified as:
\begin{itemize}
    \item buyer-initiated if $P_t > M_t$,
    \item seller-initiated if $P_t < M_t$,
    \item midpoint trades resolved using a tick-test fallback.
\end{itemize}

Despite known limitations at very high frequencies, the Lee--Ready algorithm
remains a standard benchmark for directional order flow estimation in equity
markets \citep{LeeReady1991,OddersWhite2000,Hasbrouck2007}.

\subsubsection{Daily Order Flow Imbalance}

Using classified trades, we compute daily order flow imbalance (OFI) as:
\begin{equation}
\text{OFI}_d = 
\frac{V^{\text{buy}}_d - V^{\text{sell}}_d}
     {V^{\text{buy}}_d + V^{\text{sell}}_d},
\end{equation}
where $V^{\text{buy}}_d$ and $V^{\text{sell}}_d$ denote buyer- and seller-initiated
traded volume on day $d$.

OFI captures net directional trading pressure and is closely related to the
theoretical concept of informed trading in \citet{Kyle1985} and the empirical
price impact literature \citep{Hasbrouck1991}. Normalization ensures
comparability across stocks and trading days.

\subsection{Liquidity and Spread Measures}

Liquidity conditions are measured using both quoted and effective spreads.
The quoted spread is defined as:
\begin{equation}
\text{QS}_d = \frac{A_d - B_d}{M_d},
\end{equation}
where $A_d$, $B_d$, and $M_d$ denote daily averages of the ask, bid, and midpoint.

The effective spread is computed as:
\begin{equation}
\text{ES}_d = 2 \times \frac{|P_t - M_t|}{M_t},
\end{equation}
averaged across trades.

Quoted spreads capture posted liquidity, while effective spreads capture realized
trading costs \citep{GlostenHarris1988}. Together, these measures allow us to
distinguish directional trading that occurs with minimal liquidity disruption
from trading associated with widening spreads and liquidity stress.

\subsection{Volume and Trading Intensity}

To capture trading intensity and potential liquidity shocks, we compute daily
share volume, dollar volume, and volume scaled by recent historical averages.
Abnormal volume episodes have long been associated with liquidity-driven price
pressure and temporary price dislocations
\citep{CampbellGrossmanWang1993}.

\subsection{Event-Level Feature Aggregation}

Daily microstructure measures are aggregated over the pre-announcement window
$[-10,-1]$ to form event-level summaries. Aggregation emphasizes economic
interpretability rather than statistical optimization.

Key constructs include:
\begin{itemize}
    \item \textbf{Order flow persistence}: Stability of OFI sign and magnitude
    across days, capturing gradual accumulation consistent with informed trading.
    \item \textbf{Spread stability}: Absence of systematic spread widening during
    periods of directional flow.
    \item \textbf{Volume irregularity}: Presence of large, transient volume spikes.
\end{itemize}

These constructs reflect the economic distinction between patient,
information-motivated trading and aggressive, liquidity-driven trading
\citep{Kyle1985,AdmatiPfleiderer1988}.

\subsection{Pressure Classification Framework}

For each event, we construct two latent scores:
\begin{itemize}
    \item an \emph{information pressure score}, increasing in order flow
    persistence and spread stability;
    \item a \emph{liquidity pressure score}, increasing in volume bursts and
    spread widening.
\end{itemize}

The final pressure score is defined as:
\begin{equation}
\text{Pressure}_i = \tanh(\text{InfoScore}_i - \text{LiqScore}_i),
\end{equation}
yielding a continuous measure bounded in $[-1, +1]$. Higher values indicate
information-dominated pressure, while lower values indicate liquidity-dominated
pressure. This formulation avoids arbitrary binary classification and aligns with
evidence that informed and liquidity trading coexist to varying degrees
\citep{Easley2012}.
